% Options for packages loaded elsewhere
\PassOptionsToPackage{unicode}{hyperref}
\PassOptionsToPackage{hyphens}{url}
\PassOptionsToPackage{dvipsnames,svgnames,x11names}{xcolor}
%
\documentclass[
  11pt,
]{article}

\usepackage{amsmath,amssymb}
\usepackage{iftex}
\ifPDFTeX
  \usepackage[T1]{fontenc}
  \usepackage[utf8]{inputenc}
  \usepackage{textcomp} % provide euro and other symbols
\else % if luatex or xetex
  \usepackage{unicode-math}
  \defaultfontfeatures{Scale=MatchLowercase}
  \defaultfontfeatures[\rmfamily]{Ligatures=TeX,Scale=1}
\fi
\usepackage{lmodern}
\ifPDFTeX\else  
    % xetex/luatex font selection
\fi
% Use upquote if available, for straight quotes in verbatim environments
\IfFileExists{upquote.sty}{\usepackage{upquote}}{}
\IfFileExists{microtype.sty}{% use microtype if available
  \usepackage[]{microtype}
  \UseMicrotypeSet[protrusion]{basicmath} % disable protrusion for tt fonts
}{}
\makeatletter
\@ifundefined{KOMAClassName}{% if non-KOMA class
  \IfFileExists{parskip.sty}{%
    \usepackage{parskip}
  }{% else
    \setlength{\parindent}{0pt}
    \setlength{\parskip}{6pt plus 2pt minus 1pt}}
}{% if KOMA class
  \KOMAoptions{parskip=half}}
\makeatother
\usepackage{xcolor}
\usepackage[lmargin=1in,rmargin=1in,tmargin=1in,bmargin=1in]{geometry}
\setlength{\emergencystretch}{3em} % prevent overfull lines
\setcounter{secnumdepth}{3}
% Make \paragraph and \subparagraph free-standing
\ifx\paragraph\undefined\else
  \let\oldparagraph\paragraph
  \renewcommand{\paragraph}[1]{\oldparagraph{#1}\mbox{}}
\fi
\ifx\subparagraph\undefined\else
  \let\oldsubparagraph\subparagraph
  \renewcommand{\subparagraph}[1]{\oldsubparagraph{#1}\mbox{}}
\fi


\providecommand{\tightlist}{%
  \setlength{\itemsep}{0pt}\setlength{\parskip}{0pt}}\usepackage{longtable,booktabs,array}
\usepackage{calc} % for calculating minipage widths
% Correct order of tables after \paragraph or \subparagraph
\usepackage{etoolbox}
\makeatletter
\patchcmd\longtable{\par}{\if@noskipsec\mbox{}\fi\par}{}{}
\makeatother
% Allow footnotes in longtable head/foot
\IfFileExists{footnotehyper.sty}{\usepackage{footnotehyper}}{\usepackage{footnote}}
\makesavenoteenv{longtable}
\usepackage{graphicx}
\makeatletter
\def\maxwidth{\ifdim\Gin@nat@width>\linewidth\linewidth\else\Gin@nat@width\fi}
\def\maxheight{\ifdim\Gin@nat@height>\textheight\textheight\else\Gin@nat@height\fi}
\makeatother
% Scale images if necessary, so that they will not overflow the page
% margins by default, and it is still possible to overwrite the defaults
% using explicit options in \includegraphics[width, height, ...]{}
\setkeys{Gin}{width=\maxwidth,height=\maxheight,keepaspectratio}
% Set default figure placement to htbp
\makeatletter
\def\fps@figure{htbp}
\makeatother
% definitions for citeproc citations
\NewDocumentCommand\citeproctext{}{}
\NewDocumentCommand\citeproc{mm}{%
  \begingroup\def\citeproctext{#2}\cite{#1}\endgroup}
\makeatletter
 % allow citations to break across lines
 \let\@cite@ofmt\@firstofone
 % avoid brackets around text for \cite:
 \def\@biblabel#1{}
 \def\@cite#1#2{{#1\if@tempswa , #2\fi}}
\makeatother
\newlength{\cslhangindent}
\setlength{\cslhangindent}{1.5em}
\newlength{\csllabelwidth}
\setlength{\csllabelwidth}{3em}
\newenvironment{CSLReferences}[2] % #1 hanging-indent, #2 entry-spacing
 {\begin{list}{}{%
  \setlength{\itemindent}{0pt}
  \setlength{\leftmargin}{0pt}
  \setlength{\parsep}{0pt}
  % turn on hanging indent if param 1 is 1
  \ifodd #1
   \setlength{\leftmargin}{\cslhangindent}
   \setlength{\itemindent}{-1\cslhangindent}
  \fi
  % set entry spacing
  \setlength{\itemsep}{#2\baselineskip}}}
 {\end{list}}
\usepackage{calc}
\newcommand{\CSLBlock}[1]{\hfill\break\parbox[t]{\linewidth}{\strut\ignorespaces#1\strut}}
\newcommand{\CSLLeftMargin}[1]{\parbox[t]{\csllabelwidth}{\strut#1\strut}}
\newcommand{\CSLRightInline}[1]{\parbox[t]{\linewidth - \csllabelwidth}{\strut#1\strut}}
\newcommand{\CSLIndent}[1]{\hspace{\cslhangindent}#1}

\makeatletter
\@ifpackageloaded{caption}{}{\usepackage{caption}}
\AtBeginDocument{%
\ifdefined\contentsname
  \renewcommand*\contentsname{Table of contents}
\else
  \newcommand\contentsname{Table of contents}
\fi
\ifdefined\listfigurename
  \renewcommand*\listfigurename{List of Figures}
\else
  \newcommand\listfigurename{List of Figures}
\fi
\ifdefined\listtablename
  \renewcommand*\listtablename{List of Tables}
\else
  \newcommand\listtablename{List of Tables}
\fi
\ifdefined\figurename
  \renewcommand*\figurename{Figure}
\else
  \newcommand\figurename{Figure}
\fi
\ifdefined\tablename
  \renewcommand*\tablename{Table}
\else
  \newcommand\tablename{Table}
\fi
}
\@ifpackageloaded{float}{}{\usepackage{float}}
\floatstyle{ruled}
\@ifundefined{c@chapter}{\newfloat{codelisting}{h}{lop}}{\newfloat{codelisting}{h}{lop}[chapter]}
\floatname{codelisting}{Listing}
\newcommand*\listoflistings{\listof{codelisting}{List of Listings}}
\makeatother
\makeatletter
\makeatother
\makeatletter
\@ifpackageloaded{caption}{}{\usepackage{caption}}
\@ifpackageloaded{subcaption}{}{\usepackage{subcaption}}
\makeatother
\ifLuaTeX
  \usepackage{selnolig}  % disable illegal ligatures
\fi
\usepackage{bookmark}

\IfFileExists{xurl.sty}{\usepackage{xurl}}{} % add URL line breaks if available
\urlstyle{same} % disable monospaced font for URLs
\hypersetup{
  pdftitle={Levy Income Measure of Time and Income Poverty: United States},
  pdfauthor={Fernando Rios-Avila; Aashima Sinha},
  colorlinks=true,
  linkcolor={blue},
  filecolor={Maroon},
  citecolor={Blue},
  urlcolor={Blue},
  pdfcreator={LaTeX via pandoc}}


\title{Levy Income Measure of Time and Income Poverty: United States}
\usepackage{etoolbox}
\makeatletter
\providecommand{\subtitle}[1]{% add subtitle to \maketitle
  \apptocmd{\@title}{\par {\large #1 \par}}{}{}
}
\makeatother
\subtitle{Sources, Methods and Assessment}
\author{
Fernando Rios-Avila\\
Levy Economics Institute\\
\\
\and 
Aashima Sinha\\
Levy Economics Institute\\
\\
}
\date{Invalid Date}
\begin{document}


\def\spacingset#1{\renewcommand{\baselinestretch}%
{#1}\small\normalsize} \spacingset{1}

%Ipsum lorem

\maketitle
\begin{abstract}
In this paper we present the match quality assessment of the
statistically matched data used to construct the LIMTIP estimates for
the United States, specifically for those for 2022. Using a statistical
matching procedure, we construct a synthetic dataset by combining the
American Time-Use Survey (ATUS) 2021 with Annual Social and Economic
Supplement (ASEC) 2022. First, we examine the alignment of the ATUS
(weekday and weekend) and both datasets across important demographic
characteristics. Next, we briefly describe results from the matching
algorithm. We conclude by comparing the marginal distributions of time
use between the ATUS and the synthetic dataset. Our results suggest that
our statistical matching procedure yielded a high quality of match and
the constructed synthetic dataset is appropriate for time poverty
analysis. While not presented here, similar assesments were constructed
for the ASEC matching from 2005 to 2021.
\end{abstract}
 
\vspace{.2in}

\setcounter{tocdepth}{2}



\thispagestyle{empty}
\clearpage\pagenumbering{arabic}
\newpage
\spacingset{1.2} % DON'T change the spacing!
\section{Introduction}\label{introduction}

In this paper, we describe the methodology behind the construction of
the Levy Institute Measure of Time and Income Poverty (LIMTIP) for
United States.

the construction of synthetic dataset created for use in estimation of
the Levy Institute Measure of Time and Income Poverty (LIMTIP) for
United States. The LIMTIP presents a comprehensive understanding of
households' poverty status by measuring a unique estimates of time and
income poverty combined. Construction of LIMTIP estimates requires a
variety of information at the household and individual level. In
addition to demographic characteristics, the estimation process requires
information about income and time use. However, no single datset has
access to comprehensive information of this type for all household
members across time. To produce LIMTIP estimates, we first construct a
synthetic data file by employing a statistical matching procedure to
match two source datasets: the base or recipient data (ASEC), contains
detailed demographic and income/consumption data for households and
individuals; and the donor data (ATUS) contains individuals' time
allocation in a range of daily activities including paid work, unpaid
care, domestic chores, leisure, self-care, and socializing. Using
variables that are common to both datasets we use statistical matching
to create a unique synthetic dataset that combines information from both
sources. Using this dataset, we can extract patterns of time use and
income/consumption for all household members. This in turn is useful to
study gender differences in sharing of household production among other
activities.

\section{The nature of Time Poverty and Income
Poverty}\label{sec-timepoverty}

Why do we care about Time poverty

\subsection{Accounting for Time in Poverty
Measurement}\label{accounting-for-time-in-poverty-measurement}

As expressed earlier, Poverty is a multidimensional concept that goes
beyond the simple lack of income. In addition to income, poverty can be
understood as a lack of access to resources, including time. From the
LIMPTIP perspective, Time poverty refers to the lack of time people may
have to engage on activities that are essential for taking care of the
household, its household members, self-care, and work.

As with any other measures of poverty, it is necessary to identify a
threshold that one can use to determine if the resources available by a
person or household should be classified as poor or non poor. In the
case of time, however, thinking about a threshold is less approrpiate,
because all individuals have the same amount of time available to them.

The approach that we have taken for the construction of the LIMTIP has
been to identify the time balance people face after considering the time
spent on essential activities. People with a negative balance are
considered time poor. We express the weekly time balance of individual
\(i\) in household \(j\), \(X_{ij}\), as:

\begin{equation}\phantomsection\label{eq-bal}{X_{ij} = 168 - M - \alpha_{ij}R_j-D_{ij}^0(L_{ij}+T_{ij})
}\end{equation}

where 168 is the number of hours in a week, \(M\) the sum of personal
care and nonsubstitutable household production requirements. This may
include time required for sleep, eating, personal hygiene, etc. \(R_j\)
is the required time of household production that a family \(j\) needs
to mantain the household. This may include time required for cooking,
cleaning, shopping, as well as time required for taking care of
household members, plus the time required to travel related to this
activities. Implicitly, any member of the household can engage in this
activities. \(\alpha_{ij}\) is the share of individual \(i\) in the
household production requirements. The equation also accounts for time
required for employment and commuting, where \(D_{ij}\) is the dummy
variable that takes a value of 1 if the person is employed and zero
otherwise, \(L_{ij}\) are the hours of employment, and \(T_{ij}\) the
hours of commuting.

To implement this measure, we need a dataset that contains information
on time use, in addition to standard information required for poverty
analysis. As mentioned earlier, the main source of information for time
use comes from the American Time Use Survey (ATUS), which only provides
information for a single person in the household, and a single day. As
it will be described in Section~\ref{sec-smatch}, it is necessary to
combine the ATUS with the ASEC data to construct a synthetic dataset
that contains information on time use for all household members, which
will allow us to impute all required variables for
Equation~\ref{eq-bal}.

\subsection{Time balance
Implementation.}\label{time-balance-implementation.}

Even with the synthetic dataset, it is necessary to impose some
restrictions for the proper identification of the elements in
Equation~\ref{eq-bal}. First, \(R_j\) represents the required hours that
a family with a specific household structure needs (Number of adult,
children and eldery). To better identify the requirements of a typical
household, \(R_j\) is estimated as the expected number of hours, given
the family structure, that a family with income around the poverty-level
(75-150\%), with a fall back/ and person, enganges on household
production. The requirements may reflect some extent of outsourcing of
household responsibilities. Instances of outsourcing may include
consumer purchases of substitutes that they can afford.

To caculate the actual share of household production requirements that
each individual is responsible for, we use the estimated \(R_j\)
multiplied by \(\alpha_{ij}\), which is the effective share of household
production requirements that each individual is responsible for. This is
estimated as the ratio of the individual's (imputed) time spent on
household production to the total time spent on household production by
all household members.

\(M\) represent the time required for personal care and nonsubstitutable
household production requirements, which can be typically obtained by
assumption, or using average hours based on a particular year data.

The time required for employment and commuting is calculated as total
time spend on employment and commuting, also based on the time use
survey. It may consider heterogeneity based on full and part time
employment, as well as heterogeneity across regions and years.

\subsection{Time Poverty}\label{time-poverty}

Once the time balance is calculated, time poor individuals are
identified as those with a negative time balance or time deficits. It
may useful to clarify that this time balances reflect the theoretical
deficit individuals may face under typical time requirements, thus it
does not account for how individuals may actually allocate their time.

In addition to individual time poverty status, we may also be interested
in the time poverty status of the household. To measure the poverty
status and poverty degree suffered by a household, we impose the
assumption that the deficits or superatives people face cannot be
exchanged across household members. In this case, we may consider a
household to be time poor if at least one of its members is time poor.
This is a more conservative approach, as it may be the case that some
members of the household are time poor, while others are not.

Formally,the time deficit experienced by a household \(j\) is defined
as:

\begin{equation}\phantomsection\label{eq-hhdef}{X_j = \sum_{i=1}^{j_n} \min(X_{ij},0)
}\end{equation}

where \(j_n\) is the number of individuals in the household. The time
poverty status of the household is then defined as if \(X_j < 0\), which
may indicate at least one member of the household is time poor.

\subsection{Adjusting for Income
Poverty}\label{adjusting-for-income-poverty}

Once household deficits have been calculated, it is possible to adjust
for the standard income poverty measure. This process is done by
adjusting the income poverty threshold monetazing the value of the time
deficits. The adjusted poverty lines are then used to calculate the time
and income poverty status of the household.

Call \(Z_j\) the poverty line used to determine time poverty status for
a household \(j\). The adjusted poverty that considers time poverty is
then defined as:

\begin{equation}\phantomsection\label{eq-adjpline}{Z_j^{adj} = z_j + (|X_j| * P_x)*\kappa
}\end{equation}

where \(P_x\) represents how much it would cost to buy an hour of time
(services) in the market to cover the time deficit in terms of household
production. \(\kappa\) is a parameter that adjusts the monetized value
of the time deficit to the same time unit as the poverty line. For the
case of the US-LIMTIP, we use a \(\kappa=52\), because the poverty line
is defined as an annual threshold. The new poverty threshold is then
used to calculate the time and income poverty status of the household.

\subsection{Monetazing Time Deficits}\label{monetazing-time-deficits}

How are Wages estimated. Sources of information.

\section{Statistical Matching Methodology}\label{sec-smatch}

As described in the previous section, the construction of the LIMTIP
requires having access to standard income, employment, and poverty data,
as well as time use data. While the ASEC provides detailed information
on income, employment, and demographic characteristics, in addition to
the extended measure of income and poverty, the suplemental poverty
measure (SPM), it does not have any information on time use. The ATUS,
on the other hand, provides detailed information on time use, but only
for a single individual in the household, and for a single day. To
construct the LIMTIP, it is necessary to impute the time use of all
household members, so the methodology outlined in
Section~\ref{sec-timepoverty} can be implemented.

In this section, provide a brief description of the datasets used in the
construction of the LIMTIP, and the statistical matching procedure used
to combine the datasets.

\subsection{Main Data Sources}\label{main-data-sources}

The measurement of time and income poverty requires microdata on
individuals and households with information on time spent on household
production, time spent on employment, as well as household income. Given
the importance of the intrahousehold division of labor, and the
differences across single and dual earning households, it is necessary
to have information on the time spent on household production by
households members. While good information on household production was
available in the American Time use Survey (ATUS) and good information
regarding time spent on employment and household income is available in
the Annual Social and Economic Supplement (ASEC) of the Current
Population Survey (CPS), there is no single datasource that has all
required information for the case of the US. Further more, while the
ATUS does collect detailed information on time use, it only collects
this information for a single day, and a single individual within the
household. Because of this, further efforts are necessary to create a
synthetic dataset that captures the time use of all household members,
as well as estimating the capturing a good approximation of actitivities
for a typical week, rather than a single day.

A summary of the main characteristics of the data can be found in
Table~\ref{tbl-t1}, with a brief description of ASEC and ATUS survey
data in the next subsections. Both datasets were accessed through the
Integrated Public Use Microdata Series (IPUMS) database (Flood et al.,
2023).

\begin{longtable}[]{@{}
  >{\raggedright\arraybackslash}p{(\columnwidth - 8\tabcolsep) * \real{0.2000}}
  >{\raggedright\arraybackslash}p{(\columnwidth - 8\tabcolsep) * \real{0.2000}}
  >{\raggedright\arraybackslash}p{(\columnwidth - 8\tabcolsep) * \real{0.2000}}
  >{\raggedright\arraybackslash}p{(\columnwidth - 8\tabcolsep) * \real{0.2000}}
  >{\raggedright\arraybackslash}p{(\columnwidth - 8\tabcolsep) * \real{0.2000}}@{}}
\caption{Surveys used in the constructing the LIMTIP for the United
States}\label{tbl-t1}\tabularnewline
\toprule\noalign{}
\begin{minipage}[b]{\linewidth}\raggedright
Survey
\end{minipage} & \begin{minipage}[b]{\linewidth}\raggedright
Survey Subject
\end{minipage} & \begin{minipage}[b]{\linewidth}\raggedright
Source
\end{minipage} & \begin{minipage}[b]{\linewidth}\raggedright
Sample Size
\end{minipage} & \begin{minipage}[b]{\linewidth}\raggedright
Year
\end{minipage} \\
\midrule\noalign{}
\endfirsthead
\toprule\noalign{}
\begin{minipage}[b]{\linewidth}\raggedright
Survey
\end{minipage} & \begin{minipage}[b]{\linewidth}\raggedright
Survey Subject
\end{minipage} & \begin{minipage}[b]{\linewidth}\raggedright
Source
\end{minipage} & \begin{minipage}[b]{\linewidth}\raggedright
Sample Size
\end{minipage} & \begin{minipage}[b]{\linewidth}\raggedright
Year
\end{minipage} \\
\midrule\noalign{}
\endhead
\bottomrule\noalign{}
\endlastfoot
Americal Time Use Survey & Time-Use & U.S. Census Bureau/ IPUMS & 8,136
Individuals & 2021 \\
----- & ----- & ----- & ----- & ----- \\
Annual Social and Economic Supplement (CPS-ASEC) & Income, Demographics,
Employment & U.S. Census Bureau/IPUMS & Full Sample: XXXX Restricted
Sample: 118,990 & 2022 \\
\end{longtable}

\subsubsection{Annual Social and Economic Supplement (ASEC) of the
Current Population Survey
(CPS)}\label{annual-social-and-economic-supplement-asec-of-the-current-population-survey-cps}

The CPS is a monthly survey administered by the US Bureau of Labor
Statistics. The survey collects comprehensive data on labor market
situation including statistics related to employment and unemployment,
the survey also collects detailed information on demographic
characteristics (age, sex, race, and marital status), educational
attainment, and family structure. Each household in the CPS is
interviewed for four consecutive months, not interviewed for eight, and
interviewed again for four additional months.

In March of every year, the previously interviewed households answer
additional questions, part of the ASEC supplement formerly known as the
Annual Demographic File. In addition to the basic monthly information,
this supplement provides additional data on work experience, income,
noncash benefits, and migration. In 2014, the ASEC supplement went
through a redesign of the income-collection questions. As described in
Semega and Welniak (2013), for the ASEC 2014, of the nearly 98,000
addresses in the sample, approximately one-third of the sample was
randomly assigned to be eligible to receive the redesigned income
questions. The remaining sample (approximately two-thirds) was eligible
to receive the set of ASEC income questions used in previous years,
referred to as the ``traditional income questions.'' For the statistical
matching purposes, we use the second subsample.

While we implement the statistical matching procedure for the ASEC from
2005 to 2022, the present document only describes the results for the
ASEC 2022 match. The ASEC 2022 is used as the base dataset (recipient),
as it contains rich information regarding demographics and economic
status, for every member of the household. For the matching process, we
consider data for the household, rather than family, as the aggregated
level that some demographics are constructed on. The matching, however,
is done at the individual level, considering only individuals who are
directly or indirectly related to the household head. This implies that
nonrelatives living in the household are generally not considered in the
matching process. Similarly, people not living in households, such as
those living in institutions or group quarters, are not considered in
the matching process either.

\subsubsection{American Time Use Survey
(ATUS)}\label{american-time-use-survey-atus}

The ATUS is a survey sponsored by the Bureau of Labor Statistics and
collected by the US Census Bureau. It is the first continuous survey on
time use in the United States available since 2003. Its main objective
is to provide nationally representative estimates of peoples' allocation
of time among different activities, collecting information on what they
did, where they were, and with whom they were over the course of a
single day.

The ATUS is administered to a random sample of individuals selected from
a set of eligible households that have completed their final month's
interviews for the CPS. Only one individual per household is selected to
participate in the ATUS. This individual is at least 15 years old and is
part of the civilian, noninstitutionalized population in the United
States. To be able to obtain a representative picture of time use across
one year, data collection is spread over the entire year, and
individuals are requested to report data for either a weekday or a
weekend day, but not both.

The ATUS 2021, which contains a total of \textbf{xx} observations, is
used as the donor dataset to obtain information regarding time use,
which will be transferred to the ASEC 2022.

\subsection{Statistical Matching}\label{statistical-matching}

In order to create synthetic datasets that combine data from the
different sources into a single dataset, we employ a methododology known
as statistical matching. Statistical matching is a kind of
non-parametric imputation method that allows combining information from
two independent datasets, without impossing any distributional
restrictions on the outcome variables. The basic idea of this
methodology is to combine the information from two datasets, transfering
information from one dataset (the donor) to another (the recipient). To
do this observations across surveys must be paired/linked based on how
similar (statistically similar) they are based on common and observed
characteristics, taking into acount how many individuals a survey
observation represents in the population (using Weights). Because of the
peculiarities of the ATUS, we need to implement a double matching
procedure, where each ASEC observation is matched to two ATUS
observations, one for a weekday and one for a weekend day, with the
expectation that this would better represent their typical activities
over one year, which means treating the ATUS collected for a weekday and
weekend day as two separate surveys.

The basic Statistical matching setup consists of having access to two
sources of data: survey A and survey B, which collect information from
two independent samples of the same population. Survey A collects
information \(Z, X\), whereas survey B collects information \(Z, Y\).
Although both surveys collect common information \(Z\) (for example
demographics), they each contain information on variables that are not
observed jointly: \(X\) (Consumption) and \(Y\) (Time use). In this
case, the goal of Statistical matching is to create synthetic data that
will contain all information \(Y,X, Z\), linking observations across the
datasets based on how close they are based on observed characteristics.
It is also possible to constrain matches based on the weighted
population each survey represents.

In turn, this synthetic dataset should allow researchers to analyze
otherwise unobservable relationships between, \(X\) and \(Y\), or as in
our case, income and time use (D'Orazio et al., 2006; Rässler, 2002).
Thus, inference on the relation between \(X\) and \(Y\) can only be done
to the extent that \(Z\) explains most of the common variation between
\(X\) and \(Y\).

\subsubsection{Matching Algorithm}\label{matching-algorithm}

As described in Lewaa et al. (2021), statistical matching could be
considered as a non-parametric variation of the stochastic regression
approach, where no specific distribution assumption is imposed, and the
imputed values are drawn directly from the observed distribution in the
donor file. In particular, we implement a variation of the
rank-constrained statistical matching described in Kum and Masterson
(2010), which improves on the approach by using a weight splitting
approach in combination with clustering analysis for an automatic
selection of strata groups in the approach.\footnote{ This is in
  contrast with previous iterations of Statistical matching used for the
  estimation of the LIMEW, which was based on ex-ante ad-hoc
  stratification, last-match-unit approach}

The statistical matching procedure applied for the paper is a multi-step
process that can be explained as follows:

\paragraph{Step 1: Data harmonization and weight
adjustment}\label{step-1-data-harmonization-and-weight-adjustment}

The first step involves harmonizing all common variables that will be
used in the matching process and survey balancing. This is a necessary
step in all imputation methods because variables need to have consistent
definitions before they can be utilized for imputation.

Furthermore, this step includes adjusting sample weights to ensure that
the weighted population is the same across all surveys. The standard
practice is to adjust the sample weights of the donor sample.
Additionally, for technical reasons, the weights are adjusted to be
whole numbers. While it is customary to adjust weights to match the
total population, it may also be advisable to adjust weights to align
with subpopulations based on selected strata variables.

Lastly, it is recommended to verify if both the donor and recipient
files truly represent the same population by comparing the means,
variances, and proportions of key variables across both surveys. In
instances where significant imbalances are observed, reweighted methods
can be employed to improve the balance between the surveys. However,
there is no definitive rule to determine when a discrepancy in the
distribution constitutes a substantial imbalance.

For the case of the US-LIMTIP, the ATUS sample weights are adjusted to
match the population of 15 years or older individuals in the ASEC based
on the individual \texttt{ASECWT}, excluding those not living in
households, and those not related to the head of the household. To
improve on the quality of the imputed data, we consider divide the ATUS
sample into two groups: one from weekdays and one from weekend days,
effectively matching each ASEC observation with two ATUS observations.
This is done to account for the different patterns of time use that are
observed.

For the matching implementaion, weights are adjusted for each subsample
separately, so that the weekday/weekend ATUS samples match the recipient
(ASEC) weighted population. This weights are further adjusted based on
three strata variables: gender of the respondant, if there is a child
(\textless=17 years of age) in the household, and if the respondant
declares to be working. The adjusted weights are used for the matching,
but not for the balance assessment.

\paragraph{Step 2: Strata and Cluster identification, and propensity
score
estimation}\label{step-2-strata-and-cluster-identification-and-propensity-score-estimation}

The second step involves identifying statistically similar records based
on \textbf{\emph{common}} observed characteristics. This is accomplished
through a combination of three methodologies:

\begin{itemize}
\item
  \textbf{Principal Component Analysis (PCA)}: PCA is utilized as a data
  reduction technique to decrease the dimensionality of the
  \textbf{\emph{common}} variables to a few linear combinations. While
  there are numerous suggestions on determining the optimal number of
  components, we select the first few components that explain
  approximately 50\% of the data's variation. For the construction of
  the LIMTIP, we consider variables related to household struture,
  number of children of different ages and adults of different age
  groups and gender in the household, employment status of the household
  head and spouse (if present), level of household income, house
  ownership. We also consider individual demographic characteristics
  such as age, gender, race and education level.
\item
  \textbf{Cluster analysis}: Once the principal components are
  estimated, they are employed to identify clusters within the dataset
  using a k-means cluster iterative partition algorithm. A brief
  description of the algorithm can be found in James et al. (2021).

  As this algorithm only discovers locally optimal clusters and their
  identification is influenced by random initial conditions, it has a
  tendency to generate suboptimal clusters. To mitigate this issue, we
  modify the algorithm by repeating the procedure a sufficient number of
  times and selecting the ``optimal'' cluster based on the largest
  Calinski-Harabasz pseudo-F index (Caliński and Harabasz, 1974). This
  ensures that the chosen cluster maximizes intra-cluster similarity
  while minimizing inter-cluster dissimilarity.

  This procedure generates various sets of clusters of different sizes.
  The clusters with the highest number of groups are prioritized in the
  statistical matching procedure since they represent the most similar
  records, while clusters with fewer groups are utilized in later stages
  of the matching process.
\item
  \textbf{Propensity score matching:} To enhance the matching procedure,
  we estimate a propensity score using a logit model. The dependent
  variable in the model is a binary indicator that determines whether an
  observation belongs to the donor or recipient file, while the
  independent variables consist of all common variables Z (including
  interactions or transformations). In the scenario where both surveys
  can be considered random samples from the same population, the
  expected coefficients for all variables should be zero or
  statistically insignificant. However, due to sampling variability and
  variations in survey design, it is common to observe variation in the
  propensity score. The logit model and propensity score can be
  estimated for the entire pooled survey or using the primary strata
  variables.
\end{itemize}

\subsubsection{Step 3: Matching and weight
splitting.}\label{step-3-matching-and-weight-splitting.}

Once the propensity score has been estimated and the clusters and strata
have been defined, we proceed with our matching algorithm. We start by
creating cells, which combine the identifed strata and clusters from the
previous step. The cells that combine the most detailed strata and
clusters will be used first, as they would identify the most similar
records. In addition to the Strata and clusters, this step may also
consider using other variables that are not part of the main strata, but
that are important reinforcing the similarity of the records during the
matching process.\footnote{Specifically, one variable that was used
  outside of the main Strata has been race and age, as it was found to
  be important in the matching process. These variables are only used
  for few rounds of matching, and are not part of the main strata.}

Starting with the most detailed sets of cells, records within each cell
are ranked in increasing order using the propensity score. Within each
cell, a record with the lowest propensity score from the donor file is
matched or linked to a record with the lowest propensity score in the
recipient file. If both records have the same weight, they are
considered as fully matched and removed from the donor or recipient
pool. If the weights are different, the record (donor or recipient) with
the lowest weight is removed from the pool, and the weight of the
matched record is adjusted by subtracting the weight value of the
excluded record. The record with the adjusted weight is retained in the
pool for a subsequent match.

This process of matching records and adjusting weights, if necessary,
continues until there are no more donor or recipient records left in
that cell. If there are unmatched records from the previous steps, the
procedure is repeated using a less detailed cluster until all records
from the donor and recipient files are matched. Once the matching is
completed, we obtain a synthetic dataset where all records in the donor
file are matched to potentially multiple records in the recipient files,
and vice versa. Records matched at an earlier stage are considered to be
the best matches, while those matched at later stages are considered to
be less similar.

For the final synthetic dataset, we select the ``best'' matched records
for all the donor and recipient files. In general, records that were
matched in the earlier stages (most detailed clusters) are considered to
be better than those at later stages. In case of ties, records matched
with the largest split weight are prefered. If further ties exist, the
``best'' match is randomly chosen.

Due to this step, some observations in the donor sample may not be used
at all, while others may be used more frequently than their weight would
suggest. However, if the sample sizes and weight structures across both
files are similar, we can expect only minor discrepancies between the
distribution of the imputed data in the donor and recipient datasets.
Nevertheless, if the sample sizes differ significantly, it is advisable
to use the largest file as the recipient file, which is our approach in
the US-LIMTIP construction.

The statistical matching procedure described above aims to impute all
missing values in the recipient file by transferring the observed
distribution of the imputed values from the donor file. After the
matching process is completed, and best matches are selected, we obtain
a dataset that contains unique identifiers for each record in the
recipient and donor files. This identifiers allow us to link/transfer
any information from the donor file to the recipient file. This is an
advantage over more conventional imputation methods that require a
separte imputation model for each variable.

\subsection{Estimation of Time
Deficits}\label{estimation-of-time-deficits}

\section{Statistical Matching Quality Implementation and
Assessment}\label{statistical-matching-quality-implementation-and-assessment}

\subsection{Data Alignment}\label{data-alignment}

In this section, we present the alignment of the ATUS (weekday and
weekend) and the ASEC 2022 datasets across important demographic
characteristics. The alignment is done to ensure that the two datasets
are comparable and that the matching procedure is appropriate.

** Need to create tables by Gender and all other break downs Genrder
based ** On each one do ASEC ATUS Day Atus ATUS Weekend Plust Difference

\section{Matching Procedure}\label{matching-procedure}

Briely describe the Matching and the table for matching (Rounds, number
of matches, etc)

\section{Matching Quality Assessment}\label{matching-quality-assessment}

Here we present the results for Matching Quality

\section{Conclusions}\label{conclusions}

This paper presents the quality assessment of the statistical matching
procedure used to combine the household survey data and time use data
for the United States to obtain comprehensive estimates on time and
income poverty. Overall, the two datasets are well aligned, which
warrants the implementation of the statistical matching. Based on the
statistics presented here, the matching quality is good, showing strong
balance across different household characteristics. There are, however,
a few large imbalances that are isolated in small groups. Overall, we
conclude that the statistical matching procedure employed has precisely
imputed time-use estimates for the household survey, which in turn
contributed towards producing an insightful measure of poverty for the
US-LIMTIP.

\section*{References}\label{references}
\addcontentsline{toc}{section}{References}

\phantomsection\label{refs}
\begin{CSLReferences}{1}{0}
\bibitem[\citeproctext]{ref-calinski1974}
Caliński, T., and Harabasz, J. (1974). A dendrite method for cluster
analysis. \emph{Communications in Statistics}, \emph{3}(1), 1--27.
\url{https://doi.org/10.1080/03610927408827101}

\bibitem[\citeproctext]{ref-dorazio2006}
D'Orazio, M., Di Zio, M., and Scanu, M. (2006). \emph{Statistical
{Matching}: {Theory} and {Practice}} (1st ed.). Wiley.
\url{https://doi.org/10.1002/0470023554}

\bibitem[\citeproctext]{ref-ipums2022}
Flood, S., King, M., Rodgers, R., Ruggles, S., Warren, J. R., Warren,
D., Chen, A., Cooper, G., Richards, S., Schouweiler, M., and Westberry,
M. (2023). \emph{{IPUMS}, {Current} {Population} {Survey}: {Version}
11.0}. {[}Dataset{]} Minneapolis, MN: IPUMS, 2023.
\url{https://doi.org/10.18128/D030.V11.0}

\bibitem[\citeproctext]{ref-james_2021}
James, G., Witten, D., Hastie, T., and Tibshirani, R. (2021). \emph{An
introduction to statistical learning: With applications in {R}} (Second
edition). Springer. \url{https://doi.org/10.1007/978-1-0716-1418-1}

\bibitem[\citeproctext]{ref-kum_masterson2010}
Kum, H., and Masterson, T. N. (2010). Statistical matching using
propensity scores: {Theory} and application to the analysis of the
distribution of income and wealth. \emph{Journal of Economic and Social
Measurement}, \emph{35}(3-4), 177--196.
\url{https://doi.org/10.3233/JEM-2010-0332}

\bibitem[\citeproctext]{ref-lewaa2021}
Lewaa, I., Hafez, M. S., and Ismail, M. A. (2021). Data integration
using statistical matching techniques: {A} review. \emph{Statistical
Journal of the IAOS}, \emph{37}(4), 1391--1410.
\url{https://doi.org/10.3233/SJI-210835}

\bibitem[\citeproctext]{ref-rassler2002}
Rässler, S. (2002). \emph{Statistical matching: A frequentist theory,
practical applications, and alternative {Bayesian} approaches}.
Springer.

\bibitem[\citeproctext]{ref-SemegaandWelniak2013}
Semega, J. L., and Welniak, E. (2013). \emph{Evaluating the 2013 {CPS}
{ASEC} {Income} {Redesign} {Content} {Test}}.
\url{https://www.census.gov/content/dam/Census/library/working-papers/2013/demo/semega-01.pdf}

\end{CSLReferences}



\end{document}
